% Este fichero es parte del N�mero 5 de la Revista Occam's Razor
% Revista Occam's Razor N�mero 5
%
% (c)  2010 The Occam's Razor Team
%
% Esta obra est� bajo una licencia Reconocimiento 2.5 Espa�a de Creative
% Commons. Para ver una copia de esta licencia, visite 
% http://creativecommons.org/licenses/by/2.5/es/
% o envie una carta a Creative Commons, 171 Second Street, Suite 300, 
% San Francisco, California 94105, USA.

% Este .tex contiene el contenido del articulo
% Seccion (Dejar en blanco)
%

% Incluye imagen del artículo (Debe ser diferente del Kokopelli! (°_°) Sin embargo, no es obligatoria esta imagen)



\rput(30,-30){\resizebox{!}{3.2cm}{{ \includegraphics[]{\FIGDIR/glud.png} }}}


%Imagen de el comienzo de el articulo, coordenadas desde %la parte superior izquierda del margen de la pagina
% Las imagenes quedan mejor en formato .eps, agradecería si alguien logra colocar imagenes en dicho formato y comparte la corrección.

% -------------------------------------------------
% Cabecera
\begin{flushright}
\msection{introcolor}{black}{0.35}{Seccion dejar igual}

\mtitle{10cm}{Titulo en Espa\~nol del Art\'iculo}

\msubtitle{8cm}{Titulo en Ingl\'es del Art\'iculo}

{\sf Nombre Del Autor(es), y profesi\'on }

{\psset{linecolor=black,linestyle=dotted}\psline(-12,0)}
\end{flushright}

\vspace{2mm}
% -------------------------------------------------

\begin{multicols}{2}

\sectiontext{white}{black}{Resumen}

% Introducci�n
\intro{introcolor}{E}{n esta parte por favor ubicar el Resumen de su art\'iculo, debe permitir identificar en forma r\'apida el contenido b\'asico del trabajo; no debe ser mayor a 250 palabras. No debe aportar informaci\'on o conclusi\'on que no est\'a presente en el texto, as\'i como tampoco debe  citar referencias bibliogr\'aficas. Debe quedar claro el problema que se investiga y el objetivo del mismo. 
}

\sectiontext{white}{black}{Palabras clave: } Primera, Segunda, Tercera.


\sectiontext{white}{black}{I. Introducci\'on}

A\~nadir la introducci\'on del texto. \\
Gracias a todos los que comparten su trabajo!\\
\textbf{Cualquier comentario es bien recibido.}\\
Por si alguien necesita incluir c\'odigo:


C\'odigo en Python.
\begin{verbatim}
for i in range (1, 5):
    print i
else:
    print "The for loop is over"
\end{verbatim}

C\'odigo en  C.

\begin{lstlisting}[style=C]
#include <stdio.h>
int main(int argc, char* argv[]) {
  puts("Hola mundo!");
}
\end{lstlisting}

\vspace{2mm}
 
C\'odigo en Matlab (no deber\'ia incluir esto, pero ven\'ia con la plantilla original, quiz\'a a alguien le sirva para otra cosa despu\'es).\\
\lstset{language=Matlab,frame=tb,framesep=5pt,basicstyle=\scriptsize}   
\begin{lstlisting}
function criptograma = CifrarTexto(llano,clave)
raiz = floor(sqrt(clave));
resto = clave-raiz^2;
N = length(llano);
for i=1:N,
    [raiz, resto, digito] = 
       DigitoDecimalRaizCuadrada(raiz,resto);
    cfr = llano(i)+digito;
    if cfr>90
        cfr = (cfr-91)+65;
    end
    criptograma(i) = char(cfr);
end
\end{lstlisting} 
 
\noindent

Fin de la introducci\'on.\\


\begin{entradilla}
{\em Recomendaci\'on: {\color{introcolor}{para este tipo de texto}}
  utilizarlo para resaltar lo m\'as importante o llamativo de la introducci\'on}
\end{entradilla}



Linea de texto.

\sectiontext{white}{black}{II. Metodolog\'ia}

Esta secci\'on es recomendable para los art\'iculos que contengan o hagan referencia a investigaci\'on, que incluyan experimentos.

\sectiontext{white}{black}{ Subsecci\'on de la Metodolog\'ia}

Especifica brevemente elementos y c\'omo se realiza, en caso de que pueda replicarse.


\sectiontext{white}{black}{III. Nombre de la primera Secci\'on}

Contenido del art\'iculo. \\

\textsf {\textbf {C\'odigo original }}

(Esta secci\'on solo es informativa, no se requiere que vaya en todos los art\'iculos).\\ C\'odigo tomado de la revista Occam's Razor

{\href{https://groups.google.com/forum/#!forum/revista-occams-razor}{Grupo en Google}} \\
Disponible en:  
{\href{http://www.papermint-designs.com/roor/}{Revista online Occam's razor}}

Es mejor incluir im\'agenes, quita un poco el tedio y da m\'as vida al texto para llamar la atenci\'on.

\medskip

\begin{center}
\rput(10,-10){\resizebox{!}{3.2cm}{{ \includegraphics[]{\FIGDIR/glud.png} \caption{Figura1}
\label{fig:Fig1}}}}

\end{center}

Para a\~nadir un link:
\href{https://glud.org/}{Hiperv\'inculo} \\
\href{https://blog.desdelinux.net/occamss-razor-publicacion-libre-de-divulgacion-cientifica/}{Acerca de Occam's razor} \\

% Siguiente página
%%%%%%%%%%%%%%%%%%%%%%%%%%%%%%%%%%%%%%%%%%%%%%%%%%%%%%%%%%%%
\ebOpage{introcolor}{0.35}{Seccion dejar vacio}
%%%%%%%%%%%%%%%%%%%%%%%%%%%%%%%%%%%%%%%%%%%%%%%%%%%%%%%%%%%%


Se repite la imagen anterior por flojera de cargar otra imagen, las im\'agenes que se agreguen en cada art\'iculo deben ser diferentes del Kokopelli, dado que \'este \'ultimo estar\'a presente en la portada de la revista. \\ 

Fin de la primera secci\'on.



\sectiontext{white}{black}{IV. Secci\'on con items}
Si se requiere enumeraci\'on o items:
\begin{itemize}
    \item Primero
    \item Segundo
\end{itemize}

\begin{enumerate}
    \item Primero
    \item Segundo
\end{enumerate}

Enhorabuena! 


\begin{entradilla}
{\em Este tipo de texto entradilla {\color{introcolor}{ (as\'i se define en el c\'odigo) }} 
  puede ir en cualquier parte del texto (fuera del resumen).}
\end{entradilla}

\sectiontext{white}{black}{V. Secci\'on con tablas}\\
Si se requiere incluir tablas:
\medskip


\begin{tabular}{|l|l|l|l|}
    \hline
    & \multicolumn{3}{c|}{Europa} \\
    \cline{2-4}
    & Ciudad & R\'o & S\'mbolo\\
    \hline \hline
    \multirow{3}{1cm}{Espa\~na} & Madrid & Manzanares & Cibeles\\ \cline{2-4}
    & Sevilla & Guadalquivir & Giralda\\ \cline{2-4}
    & Zaragoza & Ebro & Pilar\\ \cline{1-4}
    Francia & Par\'is & Sena & Torre Eiffel\\ \cline{1-4}
    \multirow{2}{1cm}{Italia} & Roma & T\'iber & San Pedro\\ \cline{2-4}
    & Mil\'an & \multicolumn{1}{c|}{-} & Duomo\\ \cline{1-4}
\end{tabular}

\medskip

Otros tipos de tablas:
\href{https://latexlive.files.wordpress.com/2009/04/tablas.pdf}{Tablas en \LaTeX pdf} \\


\bibliographystyle{abbrv}
\begin{bibliografia}
\bibitem{kopka}
H.~Kopka and P.~W. Daly, \emph{A Guide to \LaTeX}, 3rd~ed.\hskip 1em plus
  0.5em minus 0.4em\relax Harlow, England: Addison-Wesley, 1999.
\bibitem{kopka1}
A.~Kopka and P.~W. Daly, \emph{A Guide to \LaTeX}, 3rd~ed.\hskip 1em plus
  0.5em minus 0.4em\relax Harlow, England: Addison-Wesley, 1999.
\bibitem{kopka2}
B.~Kopka and P.~W. Daly, \emph{A Guide to \LaTeX}, 3rd~ed.\hskip 1em plus
  0.5em minus 0.4em\relax Harlow, England: Addison-Wesley, 1999.
\bibitem{kopka3}
A.~Kopka and P.~W. Daly, \emph{A Guide to \LaTeX}, 3rd~ed.\hskip 1em plus
  0.5em minus 0.4em\relax Harlow, England: Addison-Wesley, 1999.
\bibitem{kopka4}
A.~Kopka and P.~W. Daly, \emph{A Guide to \LaTeX}, 3rd~ed.\hskip 1em plus
  0.5em minus 0.4em\relax Harlow, England: Addison-Wesley, 1999.
\bibitem{kopka5}
A.~Kopka and P.~W. Daly, \emph{A Guide to \LaTeX}, 3rd~ed.\hskip 1em plus
  0.5em minus 0.4em\relax Harlow, England: Addison-Wesley, 1999.
\bibitem{kopka6}
A.~Kopka and P.~W. Daly, \emph{A Guide to \LaTeX}, 3rd~ed.\hskip 1em plus
  0.5em minus 0.4em\relax Harlow, England: Addison-Wesley, 1999.
\end{bibliografia}


\begin{biografia}{}{Nombre Completo del Autor} 
% añadir fotografía tamaño [2.5 cm x 3.3 cm ] 
%  |¬_¬|'  La foto es Opcional! , se añade reemplazando:
% \begin{biografia}{images/mi_articulo/autor.eps}{Nombre Completo del Autor}


Corta descripci\'on del autor, que incluya estudios de pregrado o postgrado, pertenencia a grupos de investigaci\'on
\end{biografia}



\raggedcolumns
\pagebreak


\end{multicols}

\clearpage
\pagebreak
