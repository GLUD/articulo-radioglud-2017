% Este fichero es parte del Número 5 de la Revista Occam's Razor
% Revista Occam's Razor Número 5
%
% (c)  2010 The Occam's Razor Team
%
% Esta obra está bajo una licencia Reconocimiento 2.5 España de Creative
% Commons. Para ver una copia de esta licencia, visite 
% http://creativecommons.org/licenses/by/2.5/es/
% o envie una carta a Creative Commons, 171 Second Street, Suite 300, 
% San Francisco, California 94105, USA.

% Este .tex contiene el contenido del articulo
% Seccion (Dejar en blanco)
%

% Incluye imagen del artículo (Debe ser diferente del Kokopelli! (°_°) Sin embargo, no es obligatoria esta imagen)

\rput(2.5,-2.3){\resizebox{!}{3.2cm}{{\includegraphics[]{images/nombre_articulo/glud}}}}


%Imagen de el comienzo de el articulo, coordenadas desde %la parte superior izquierda del margen de la pagina
% Las imagenes quedan mejor en formato .eps, agradecería si alguien logra colocar imagenes en dicho formato y comparte la corrección.

% -------------------------------------------------
% Cabecera
\begin{flushright}
\msection{introcolor}{black}{0.35}{Seccion dejar igual}

\mtitle{10cm}{Titulo en Español del Artículo}

\msubtitle{8cm}{Titulo en Inglés del Artículo}

{\sf Nombre Del Autor(es), y profesión }

{\psset{linecolor=black,linestyle=dotted}\psline(-12,0)}
\end{flushright}

\vspace{2mm}
% -------------------------------------------------

\begin{multicols}{2}

\sectiontext{white}{black}{Resumen}

% Introducción
\intro{introcolor}{E}{n esta parte por favor ubicar el Resumen de su artículo, debe permitir identificar en forma rápida el contenido básico del trabajo; no debe ser mayor a 250 palabras. No debe aportar información o conclusión que no está presente en el texto, así como tampoco debe  citar referencias bibliográficas. Debe quedar claro el problema que se investiga y el objetivo del mismo. 
}

\sectiontext{white}{black}{Palabras clave: } Primera, Segunda, Tercera.


\sectiontext{white}{black}{I. Introducción}

Añadir la introducción del texto. \\
Gracias a todos los que comparten su trabajo!\\
\textbf{Cualquier comentario es bien recibido.}\\
Por si alguien necesita incluir código:


Código en Python.
\begin{verbatim}
for i in range (1, 5):
    print i
else:
    print "The for loop is over"
\end{verbatim}

Código en  C.

\begin{lstlisting}
#include <stdio.h>
int main(int argc, char* argv[]) {
  puts("Hola mundo!");
}
\end{lstlisting}

\vspace{2mm}
 
Código en Matlab (no debería incluir esto, pero venía con la plantilla original, quizá a alguien le sirva para otra cosa después).\\
\lstset{language=Matlab,frame=tb,framesep=5pt,basicstyle=\scriptsize}   
\begin{lstlisting}
function criptograma = CifrarTexto(llano,clave)
raiz = floor(sqrt(clave));
resto = clave-raiz^2;
N = length(llano);
for i=1:N,
    [raiz, resto, digito] = 
       DigitoDecimalRaizCuadrada(raiz,resto);
    cfr = llano(i)+digito;
    if cfr>90
        cfr = (cfr-91)+65;
    end
    criptograma(i) = char(cfr);
end
\end{lstlisting} 
 
\noindent

Fin de la introducción.\\


\begin{entradilla}
{\em Recomendación: {\color{introcolor}{para este tipo de texto}}
  utilizarlo para resaltar lo más importante o llamativo de la introducción}
\end{entradilla}



Linea de texto.

\sectiontext{white}{black}{II. Metodología}

Esta sección es recomendable para los artículos que contengan o hagan referencia a investigación, que incluyan experimentos.

\sectiontext{white}{black}{ Subsección de la Metodología}

Especifica brevemente elementos y cómo se realiza, en caso de que pueda replicarse.


\sectiontext{white}{black}{III. Nombre de la primera Sección}

Contenido del artículo. \\

\textsf {\textbf {Código original }}

(Esta sección solo es informativa, no se requiere que vaya en todos los artículos).\\ Código tomado de la revista Occam's Razor

{\href{https://groups.google.com/forum/#!forum/revista-occams-razor}{Grupo en Google}} \\
Disponible en:  
{\href{http://www.papermint-designs.com/roor/}{Revista online Occam's razor}}

Es mejor incluir imágenes, quita un poco el tedio y da más vida al texto para llamar la atención.

\begin{center}
\resizebox{5cm}{!}{{\epsfbox{images/nombre_articulo/glud.eps}}}
\mycaption{Gráfico del movimiento de los datos multimedia desde la generación del fuente hasta que lo escucha el usuario o oyente de la radio o canal de vídeo}
\end{center}

% Siguiente página
%%%%%%%%%%%%%%%%%%%%%%%%%%%%%%%%%%%%%%%%%%%%%%%%%%%%%%%%%%%%
\ebOpage{introcolor}{0.35}{Seccion dejar vacio}
%%%%%%%%%%%%%%%%%%%%%%%%%%%%%%%%%%%%%%%%%%%%%%%%%%%%%%%%%%%%

Para añadir un link:
\href{https://glud.org/}{Hipervínculo} \\
\href{https://blog.desdelinux.net/occamss-razor-publicacion-libre-de-divulgacion-cientifica/}{Acerca de Occam's razor} \\


Se repite la imagen anterior por flojera de cargar otra imagen, las imágenes que se agreguen en cada artículo deben ser diferentes del Kokopelli, dado que éste último estará presente en la portada de la revista. \\ 

Fin de la primera sección.



\sectiontext{white}{black}{IV. Sección con items}
Si se requiere enumeración o items:
\begin{itemize}
    \item Primero
    \item Segundo
\end{itemize}

\begin{enumerate}
    \item Primero
    \item Segundo
\end{enumerate}

Enhorabuena! 


\begin{entradilla}
{\em Este tipo de texto entradilla {\color{introcolor}{ (así se define en el código) }} 
  puede ir en cualquier parte del texto (fuera del resumen).}
\end{entradilla}

\sectiontext{white}{black}{V. Sección con tablas}\\
Si se requiere incluir tablas:
\medskip


\begin{tabular}{|l|l|l|l|}
    \hline
    & \multicolumn{3}{c|}{Europa} \\
    \cline{2-4}
    & Ciudad & Ró & S\'mbolo\\
    \hline \hline
    \multirow{3}{1cm}{España} & Madrid & Manzanares & Cibeles\\ \cline{2-4}
    & Sevilla & Guadalquivir & Giralda\\ \cline{2-4}
    & Zaragoza & Ebro & Pilar\\ \cline{1-4}
    Francia & París & Sena & Torre Eiffel\\ \cline{1-4}
    \multirow{2}{1cm}{Italia} & Roma & Tíber & San Pedro\\ \cline{2-4}
    & Milán & \multicolumn{1}{c|}{-} & Duomo\\ \cline{1-4}
\end{tabular}

\medskip

Otros tipos de tablas:
\href{https://latexlive.files.wordpress.com/2009/04/tablas.pdf}{Tablas en \LaTeX pdf} \\


\bibliographystyle{abbrv}
\begin{bibliografia}
\bibitem{kopka}
H.~Kopka and P.~W. Daly, \emph{A Guide to \LaTeX}, 3rd~ed.\hskip 1em plus
  0.5em minus 0.4em\relax Harlow, England: Addison-Wesley, 1999.
\bibitem{kopka1}
A.~Kopka and P.~W. Daly, \emph{A Guide to \LaTeX}, 3rd~ed.\hskip 1em plus
  0.5em minus 0.4em\relax Harlow, England: Addison-Wesley, 1999.
\bibitem{kopka2}
B.~Kopka and P.~W. Daly, \emph{A Guide to \LaTeX}, 3rd~ed.\hskip 1em plus
  0.5em minus 0.4em\relax Harlow, England: Addison-Wesley, 1999.
\bibitem{kopka3}
A.~Kopka and P.~W. Daly, \emph{A Guide to \LaTeX}, 3rd~ed.\hskip 1em plus
  0.5em minus 0.4em\relax Harlow, England: Addison-Wesley, 1999.
\bibitem{kopka4}
A.~Kopka and P.~W. Daly, \emph{A Guide to \LaTeX}, 3rd~ed.\hskip 1em plus
  0.5em minus 0.4em\relax Harlow, England: Addison-Wesley, 1999.
\bibitem{kopka5}
A.~Kopka and P.~W. Daly, \emph{A Guide to \LaTeX}, 3rd~ed.\hskip 1em plus
  0.5em minus 0.4em\relax Harlow, England: Addison-Wesley, 1999.
\bibitem{kopka6}
A.~Kopka and P.~W. Daly, \emph{A Guide to \LaTeX}, 3rd~ed.\hskip 1em plus
  0.5em minus 0.4em\relax Harlow, England: Addison-Wesley, 1999.
\end{bibliografia}


\begin{biografia}{images/nombre_articulo/glud.eps}{Nombre Completo del Autor} 
% añadir fotografía tamaño [2.5 cm x 3.3 cm ] 
%  |¬_¬|'  La foto es Opcional! , se añade reemplazando:
% \begin{biografia}{images/mi_articulo/autor.eps}{Nombre Completo del Autor}
Corta descripción del autor, que incluya estudios de pregrado o postgrado, pertenencia a grupos de investigación
\end{biografia}

\raggedcolumns
\pagebreak


\end{multicols}

\clearpage
\pagebreak
